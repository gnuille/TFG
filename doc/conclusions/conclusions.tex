\chapter{Conclusions}

With this final chapter, we close the work. The main takeaways from the study are:
\begin{itemize}

  \item We helped the module developers to have a better understanding of their code performance.
  \item We achieved a relevant speedup against the vanilla version. It will allow combustion researchers to get results faster and use fewer resources. 
  \item This study has motivated further work on analyzing in-depth the Load Balance and getting the metrics at run-time using Tracking Application Life Performance (TALP), a DLB module. We have submitted (pending of revision and acceptance) a study to the International Supercomputing Conference, presenting TALP and demonstrating it using this Alya module.
\end{itemize}

On the personal side:
\begin{itemize}
  \item This project has introduced me to research, precisely on the high-performance computing field.
  \item I have developed in-depth my performance analysis skills.
  \item Working in a project in conjunction with non-experts in the performance analysis field has given me experience in teamwork and communication skills.
  \item This project has introduced into writing academic articles.
  \item Applying OmpSs and DLB to the code has made me grown my technical and learning new things skills.
  \item Developing the analysis have grown my working methodology, specifically on managing data and version control of code.
\end{itemize}

\section{Future work}

We planned a contribution to the combustion domain's scientific community, presenting the approach to address load imbalance in the combustion using the optimization presented in this work.  
