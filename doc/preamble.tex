\usepackage[utf8]{inputenc}
\usepackage{calc}                %% Allows basic operations inside LaTeX commands
\usepackage{pslatex}             %% Arialfont 
\usepackage{fancyhdr,nextpage}   %% Fancy Headers. Next page can force empty style when  
\usepackage[T1]{fontenc}
\usepackage{helvet}              %% Set the default font family
\renewcommand{\familydefault}{\sfdefault}  %% sans-serif is default
\usepackage{microtype}
\usepackage{graphicx}
\usepackage{caption}
\usepackage{subcaption}
\usepackage[UKenglish]{babel}
\usepackage[colorlinks,bookmarksopen,bookmarksnumbered,citecolor=black,urlcolor=blue]{hyperref}
\usepackage{tracedraw}
\usepackage{array}
\usepackage{siunitx}
\usepackage{rotating}
\usepackage{eurosym}
\usepackage{xcolor}
\usepackage{listings}
\usepackage{booktabs}
\usepackage{tikz}
\usetikzlibrary{trees,positioning,arrows}
\usepackage{amsmath,amssymb,amsfonts}
\usepackage{colortbl}
\usepackage{lipsum}
\usepackage{lmodern}
\usepackage[most]{tcolorbox}

% 
%-------------- equations stuff
\newcommand{\mybm}[1]    {\mbox{\boldmath{$#1$}}}
\newcommand{\boldu}{{\mybm u}}
\newcommand{\defor}    {{\mybm \varepsilon}}
\newcommand{\boldo}    {{\mybm 0}}
\newcommand{\boldg}    {{\mybm g}}
\newcommand{\boldtau}  {{\mybm \tau}}
%--------------- end equations stuff

\lstset{
    frame=tb, % draw a frame at the top and bottom of the code block
    tabsize=4, % tab space width
    showstringspaces=false, % don't mark spaces in strings
    numbers=left, % display line numbers on the left
    commentstyle=\color{gray}, % comment color
    keywordstyle=\color{blue}, % keyword color
    stringstyle=\color{red}, % string color
    emph={loops, loads, duration},      emphstyle=\color{red},
    basicstyle=\footnotesize\ttfamily,
    breaklines=true,
    captionpos=b
}

\setlength{\oddsidemargin}{0.46cm}      %   Left margin on odd-numbered pages.
\setlength{\evensidemargin}{0.46cm}       %   Left margin on even-numbered pages.
\setlength{\marginparwidth}{0cm}        %   Width of marginal notes.
\setlength{\marginparsep}{0pt}            % Horizontal space between outer margin and marginal note

\graphicspath{ {./graphics/} }

\setlength{\headheight}{1\baselineskip + 0.1pt}  %    Height of box containing running head.
\setlength{\headsep}{8mm}                %    Space between running head and text.
\setlength{\topmargin}{-0.04cm - \headheight - \headsep}          
                                         %    Nominal distance from top of page to top of box containing running head.
% Bottom of page:
\setlength{\footskip}{7mm}       %    Distance from baseline of box containing foot to baseline of last line of text.

% DIMENSION OF TEXT
%%%%%%%%%%%%%%%%%%%

\setlength{\textheight}{24.6cm}    % \textheight is the height of text 
                                   %(including footnotes and figures, excluding running head and foot).
\setlength{\textwidth}{15cm}       % Width of text line.

% A \raggedbottom command causes 'ragged bottom' pages: pages set to
% natural height instead of being stretched to exactly \textheight.
\raggedbottom

% FOOTNOTES
%%%%%%%%%%%
\setlength{\footnotesep}{8.4pt}    % Height of strut placed at the beginning of every footnote = 
                                   % height of normal \footnotesize strut, so no extra space between footnotes.
%%\skip\footins 10.8pt plus 4pt minus 2pt  % Space between last line of text and 
                                         % top of first footnote.

% FLOATS: (a float is something like a figure or table)
%%%%%%%%%%%%%%%%%%%%%%%%%%%%%%%%%%%%%%%%%%%%%%%%%%%%%%%

% FOR FLOATS ON A TEXT PAGE
%    ONE-COLUMN MODE OR SINGLE-COLUMN FLOATS IN TWO-COLUMN MODE:
\setlength{\floatsep}{14pt plus 2pt minus 4pt}     % Space between adjacent floats moved  to top or bottom of text page.
\setlength{\textfloatsep}{20pt plus 2pt minus 4pt} % Space between main text and floats at top or bottom of page.
\setlength{\intextsep}{14pt plus 4pt minus 4pt}    % Space between in-text figures and text
%    TWO-COLUMN FLOATS IN TWO-COLUMN MODE:
\setlength{\dblfloatsep}{14pt plus 2pt minus 4pt}  % Same as \floatsep for double-column figures in two-column mode.
\setlength{\dbltextfloatsep}{20pt plus 2pt minus 4pt}% \textfloatsep for double-column floats.


% MARGINAL NOTES
%%%%%%%%%%%%%%%%
\setlength{\marginparpush}{7pt}       % Minimum vertical separation between two marginal notes.

\DeclareUnicodeCharacter{2514}{\hookrightarrow}
