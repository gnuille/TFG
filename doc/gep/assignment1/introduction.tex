\section{Introduction}

Computational science is a growing field that allows researchers to predict the behaviour of systems. For example, a physicist can write a program to predict the trajectory of a sphere instead of calculating it by hand.

 What if we bring this further?  We can try to optimize the aero-dynamism of a plane to save fuel, the behaviour of the human body cells to a recently designed drug or even given a genome figure out the risk of cancer.

Computing these are at a high cost. It is not conceivable to solve the problems on our laptops as the number of calculations and data required to process is overwhelming. For solving these tasks, we need supercomputers.

Supercomputers are machines with a huge compute power oriented to scientific and technique jobs. These machines are many small machines interconnected. Developers must write their code in a manner that the program is capable of run in parallel. 

Writing efficient parallel programs is a difficult task and even more so if the developer is not specialized in computer science. In this work, we will study and improve an existing application that simulates combustion processes.

