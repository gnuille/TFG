\section{Efficiency metrics}

For gathering the efficiency metrics of the application, we need traces. For the study, we ran with 8, 48, 96, 384 and 768 processes. Table \ref{modelfactors} shows the efficiency metrics for the detailed case. We can see that parallel efficiency in the base case (8 processes) is 75\%, and this tells us the application scalability is terrible and is far from being able to scale to a high number of cores.  The main limiting factor is the load balance going from 75\% to 35\%. The other two limiting factors are the transfer efficiency and the instruction scalability. These two factors will determine the performance when the load balance is solved.


\begin{table}[htbp]
\centering
\setlength{\extrarowheight}{0pt}
\addtolength{\extrarowheight}{\aboverulesep}
\addtolength{\extrarowheight}{\belowrulesep}
\setlength{\aboverulesep}{0pt}
\setlength{\belowrulesep}{0pt}
\begin{tabular}{|l|l|l|l|l|l|} 
\toprule
\textbf{Number of processes}    & \textbf{8}                                 & \textbf{48}                               & \textbf{96}                               & \textbf{384}                              & \textbf{768}                               \\ 
\hline
-Parallel efficiency            & {\cellcolor[rgb]{0.859,0.341,0.341}}75.63  & {\cellcolor[rgb]{0.859,0.341,0.341}}59.10 & {\cellcolor[rgb]{0.859,0.341,0.341}}49.40 & {\cellcolor[rgb]{0.859,0.341,0.341}}36.14 & {\cellcolor[rgb]{0.859,0.341,0.341}}26.28  \\ 
\hline
~ -Load balance                 & {\cellcolor[rgb]{0.859,0.341,0.341}}75.82  & {\cellcolor[rgb]{0.859,0.341,0.341}}59.99 & {\cellcolor[rgb]{0.859,0.341,0.341}}50.83 & {\cellcolor[rgb]{0.859,0.341,0.341}}41.71 & {\cellcolor[rgb]{0.859,0.341,0.341}}35.75  \\ 
\hline
~ -Communication efficiency     & {\cellcolor[rgb]{0.408,0.753,0.486}}99.74  & {\cellcolor[rgb]{0.478,0.773,0.49}}98.53  & {\cellcolor[rgb]{0.565,0.796,0.494}}97.18 & {\cellcolor[rgb]{0.988,0.749,0.482}}86.65 & {\cellcolor[rgb]{0.859,0.341,0.341}}73.52  \\ 
\hline
~~~~~ -Serialization efficiency & {\cellcolor[rgb]{0.4,0.749,0.486}}99.86    & {\cellcolor[rgb]{0.431,0.761,0.486}}99.32 & {\cellcolor[rgb]{0.443,0.761,0.486}}99.13 & {\cellcolor[rgb]{0.643,0.82,0.498}}95.83  & {\cellcolor[rgb]{0.996,0.89,0.51}}89.39    \\ 
\hline
~~~~~ -Transfer efficiency      & {\cellcolor[rgb]{0.396,0.749,0.486}}99.88  & {\cellcolor[rgb]{0.439,0.761,0.486}}99.21 & {\cellcolor[rgb]{0.51,0.78,0.49}}98.03    & {\cellcolor[rgb]{0.976,0.918,0.518}}90.42 & {\cellcolor[rgb]{0.976,0.525,0.439}}82.25  \\ 
\hline
~~ -Computation scalability     & {\cellcolor[rgb]{0.388,0.745,0.482}}100.00 & {\cellcolor[rgb]{0.918,0.898,0.514}}91.37 & {\cellcolor[rgb]{0.996,0.902,0.514}}89.62 & {\cellcolor[rgb]{0.976,0.525,0.439}}83.80 & {\cellcolor[rgb]{0.859,0.341,0.341}}78.30  \\ 
\hline
~~~~~ -IPC scalability          & {\cellcolor[rgb]{0.388,0.745,0.482}}100.00 & {\cellcolor[rgb]{0.824,0.871,0.51}}93.29  & {\cellcolor[rgb]{0.824,0.871,0.51}}93.55  & {\cellcolor[rgb]{0.824,0.871,0.51}}93.36  & {\cellcolor[rgb]{0.824,0.871,0.51}}92.90   \\ 
\hline
~~~~~ -Instruction scalability  & {\cellcolor[rgb]{0.388,0.745,0.482}}100.00 & {\cellcolor[rgb]{0.537,0.788,0.494}}97.61 & {\cellcolor[rgb]{0.647,0.82,0.498}}95.79  & {\cellcolor[rgb]{0.996,0.906,0.514}}89.75 & {\cellcolor[rgb]{0.976,0.525,0.439}}84.27  \\
\bottomrule
\end{tabular}
\caption[Efficiency metrics.]{Efficiency metrics. Own compilation.}
\label{modelfactors}
\end{table}


We can observe that the IPC scalability lowers when we jump from 8 processes to 48 it stabilizes. This happens because we are filling the nodes and the memory bandwidth, the caches, and in general, the node resources saturate. 

As we saw in the \textit{focus of analysis} section, the main limiting factor is the load balance. This issue is probably due to work not being adequately distributed among processes. We will examine this flaw in the next section.

The transfer and serialization efficiencies are also an issue we need to focus on, the application scaling to a higher number of cores will lose all the efficiency due to communications, although the load balance is addressed.
